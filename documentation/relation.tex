% Preamble
\documentclass[11pt]{article}

% Packages
\usepackage{amsmath}
\usepackage[italian]{babel}

% Document
\title{Utilizzo di OpenMP in problemi di render
Parallel Programming for Machine Learning}
\author{Lorenzo Baiardi, Thomas Del Moro}
\date{GG MM AAAA}
\begin{document}

    \maketitle
        \clearpage

    \tableofcontents
        \clearpage

    \section{Introduzione}\label{sec:introduzione}
        In questo elaborato vogliamo dimostrare l'efficacia della parallelizzazione nel campo della computazione di
        problemi qualsiasi.
        \clearpage

    \section{Analisi del problema}\label{sec:analisi-del-problema}
        Il problema che andremo a testare consiste nel effettuare un render di cerchi su un numero specificato di piani.
        \clearpage

    \section{Parallelizzazione}\label{sec:parallelizazzione}
        Qui riporteremo le operazioni di parallelizzazione che abbiamo eseguito all'interno del progetto.
        Per la parallelizzazione ci siamo affidati alla libreria OPENMP .
        \clearpage

    \section{Test}\label{sec:test}
        Abbiamo eseguito il test per N volte, variando sia il numero di piani da sommare che nel numero di cerchi
        da disegnare su ciascun piano.
        I cerchi generati solo gli stessi sia nel caso del problema sequenziale che nel problema parallelo.
        @Todo DATI .
        \clearpage

    \section{Conclusioni}\label{sec:conclusioni}
        Valutazione dei tempi, speedup, etc.

\end{document}